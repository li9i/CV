\documentclass[a4paper,10pt,twoside]{article}
\usepackage{fullpage}
\usepackage{fontspec}
\usepackage[utf8]{inputenc}

%\defaultfontfeatures{Mapping=tex-text}
%\setmainfont[Mapping=tex-text]{GFS Didot} % Main document font
%\setmainfont{cmr12} % Main document font
\setmainfont{GFSdidot} % Main document font

\usepackage[usenames,dvipsnames]{xcolor} % Required for specifying custom colors

%\usepackage[big]{layaureo} % Margin formatting of the A4 page, an alternative to layaureo can be
\usepackage{fullpage}
% To reduce the height of the top margin uncomment:
\addtolength{\voffset}{-1.3cm}

\usepackage{hyperref} % Required for adding links	and customizing them
\definecolor{linkcolour}{rgb}{0,0.2,0.6} % Link color
\hypersetup{colorlinks,breaklinks,urlcolor=linkcolour,linkcolor=linkcolour} % Set link colors throughout the document

\usepackage{titlesec} % Used to customize the \section command
\titleformat{\section}{\Large\scshape\raggedright}{}{0em}{}[\titlerule] % Text formatting of sections
\titlespacing{\section}{0pt}{3pt}{3pt} % Spacing around sections


\usepackage{fancyhdr}
\setlength{\headheight}{15.2pt}
\setlength{\textheight}{710pt}
\pagestyle{fancy}
\fancyhead[]{}
\renewcommand{\headrulewidth}{0pt}
%\fancyfoot[LO,RE]{Βιογραφικό Σημείωμα $-$ Αλέξανδρος Φιλοθέου}
\fancyfoot[LO,RE]{}

\usepackage{wasysym}
\usepackage{fontawesome}
\usepackage{tikz}

\usepackage[backend=biber,style=alphabetic,sorting=ynt,babel=other,bibencoding=utf8, language=autobib, maxnames=99]{biblatex}
\addbibresource{papers_during_phd.bib}
\begin{filecontents}{papers_during_phd.bib}
@article{Tzitzis2020,
author = {Tzitzis, Anastasios and Megalou, Spyros and Siachalou, Stavroula and Emmanouil, Tsardoulias G. and Filotheou, Alexandros and Yioultsis, Traianos V. and Dimitriou, Antonis G.},
doi = {10.1109/JRFID.2020.3000332},
issn = {2469-7281},
journal = {IEEE Journal of Radio Frequency Identification},
month = {dec},
number = {4},
pages = {283--299},
title = {{Trajectory Planning of a Moving Robot Empowers 3D Localization of RFID Tags With a Single Antenna}},
url = {https://ieeexplore.ieee.org/document/9109328/},
volume = {4},
year = {2020}
}
@inproceedings{Tzitzis2020a,
author = {Tzitzis, Anastasios and Filotheou, Alexandros and Siachalou, Stavroula and Tsardoulias, Emmanouil and Megalou, Spyros and Bletsas, Aggelos and Panayiotou, Konstantinos and Symeonidis, Andreas and Yioultsis, Traianos and Dimitriou, Antonis G.},
booktitle = {2020 IEEE International Conference on RFID (RFID)},
doi = {10.1109/RFID49298.2020.9244904},
isbn = {978-1-7281-5576-0},
month = {sep},
pages = {1--8},
publisher = {IEEE},
title = {{Real-time 3D localization of RFID-tagged products by ground robots and drones with commercial off-the-shelf RFID equipment: Challenges and Solutions}},
url = {https://ieeexplore.ieee.org/document/9244904/},
year = {2020}
}
@inproceedings{Mylonopoulos2021,
author = {Mylonopoulos, George and Chatzistefanou, Aristidis Raptopoulos and Filotheou, Alexandros and Tzitzis, Anastasios and Siachalou, Stavroula and Dimitriou, Antonis G.},
booktitle = {2021 IEEE International Conference on RFID Technology and Applications (RFID-TA)},
doi = {10.1109/RFID-TA53372.2021.9617436},
isbn = {978-1-6654-2657-2},
month = {oct},
pages = {32--35},
publisher = {IEEE},
title = {{Localization, Tracking and Following a Moving Target by an RFID Equipped Robot}},
url = {https://ieeexplore.ieee.org/document/9617436/},
year = {2021}
}
@inproceedings{Dimitriou2021a,
author = {Dimitriou, Antonis and Tzitzis, Anastasios and Filotheou, Alexandros and Megalou, Spyros and Siachalou, Stavroula and Chatzistefanou, Aristidis R. and Malama, Andreana and Tsardoulias, Emmanouil and Panayiotou, Konstantinos and Giannelos, Evaggelos and Vasiliadis, Thodoris and Mouroutsos, Ioannis and Karanikas, Ioannis and Petrou, Loukas and Symeonidis, Andreas and Sahalos, John and Yioultsis, Traianos and Bletsas, Aggelos},
booktitle = {2021 6th International Conference on Smart and Sustainable Technologies (SpliTech)},
doi = {10.23919/SpliTech52315.2021.9566425},
isbn = {978-953-290-112-2},
month = {sep},
pages = {01--06},
publisher = {IEEE},
title = {{Autonomous Robots, Drones and Repeaters for Fast, Reliable, Low-Cost RFID Inventorying {\&} Localization}},
url = {https://ieeexplore.ieee.org/document/9566425/},
year = {2021}
}
@article{Tzitzis2023,
author = {Tzitzis, Anastasios and Filotheou, Alexandros and Chatzistefanou, Aristidis Raptopoulos and Yioultsis, Traianos and Dimitriou, Antonis G.},
doi = {10.1109/JRFID.2023.3288982},
issn = {2469-7281},
journal = {IEEE Journal of Radio Frequency Identification},
pages = {1--1},
title = {{Real-Time Global Localization of a Mobile Robot by Exploiting RFID Technology}},
url = {https://ieeexplore.ieee.org/document/10160120/},
year = {2023}
}
@INPROCEEDINGS{8739423,
  author={Tzitzis, Anastasios and Megalou, Spyros and Siachalou, Stavroula and Yioultsis, Traianos and Kehagias, Athanasios and Tsardoulias, Emmanouil and Filotheou, Alexandros and Symeonidis, Andreas and Petrou, Loukas and Dimitriou, Antonis G.},
  booktitle={2019 13th European Conference on Antennas and Propagation (EuCAP)},
  title={Phase ReLock - Localization of RFID Tags by a Moving Robot},
  year={2019},
  volume={},
  number={},
  pages={1-5},
  doi={}}
@INPROCEEDINGS{8739486,
  author={Megalou, Spyros and Tzitzis, Anastasios and Siachalou, Stavroula and Yioultsis, Traianos and Sahalos, John and Tsardoulias, Emmanouil and Filotheou, Alexandros and Symeonidis, Andreas and Petrou, Loukas and Bletsas, Aggelos and Dimitriou, Antonis G.},
  booktitle={2019 13th European Conference on Antennas and Propagation (EuCAP)},
  title={Fingerprinting Localization of RFID tags with Real-Time Performance-Assessment, using a Moving Robot},
  year={2019},
  volume={},
  number={},
  pages={1-5},
  doi={}}
@article{Filotheou2020bA,
author = {Filotheou, Alexandros and Tsardoulias, Emmanouil and Dimitriou, Antonis and Symeonidis, Andreas and Petrou, Loukas},
doi = {10.1007/s10846-019-01086-y},
issn = {0921-0296},
journal = {Journal of Intelligent {\&} Robotic Systems},
month = {jun},
number = {3-4},
pages = {567--601},
title = {{Quantitative and Qualitative Evaluation of ROS-Enabled Local and Global Planners in 2D Static Environments}},
url = {http://link.springer.com/10.1007/s10846-019-01086-y},
volume = {98},
year = {2020}
}
@article{Filotheou2020cA,
author = {Filotheou, Alexandros and Tsardoulias, Emmanouil and Dimitriou, Antonis and Symeonidis, Andreas and Petrou, Loukas},
doi = {10.1007/s10846-020-01253-6},
issn = {0921-0296},
journal = {Journal of Intelligent {\&} Robotic Systems},
month = {dec},
number = {3-4},
pages = {925--944},
title = {{Pose Selection and Feedback Methods in Tandem Combinations of Particle Filters with Scan-Matching for 2D Mobile Robot Localisation}},
url = {https://link.springer.com/10.1007/s10846-020-01253-6},
volume = {100},
year = {2020}
}
@article{Filotheou2022eA,
author = {Filotheou, Alexandros and Tzitzis, Anastasios and Tsardoulias, Emmanouil and Dimitriou, Antonis and Symeonidis, Andreas and Sergiadis, George and Petrou, Loukas},
doi = {10.1007/s10846-021-01535-7},
issn = {0921-0296},
journal = {Journal of Intelligent {\&} Robotic Systems},
month = {feb},
number = {2},
pages = {26},
title = {{Passive Global Localisation of Mobile Robot via 2D Fourier-Mellin Invariant Matching}},
url = {https://link.springer.com/10.1007/s10846-021-01535-7},
volume = {104},
year = {2022}
}
@article{Filotheou2022,
author = {Filotheou, Alexandros},
doi = {10.1016/j.robot.2021.103957},
issn = {09218890},
journal = {Robotics and Autonomous Systems},
month = {mar},
pages = {103957},
title = {{Correspondenceless scan-to-map-scan matching of homoriented 2D scans for mobile robot localisation}},
url = {https://linkinghub.elsevier.com/retrieve/pii/S0921889021002323},
volume = {149},
year = {2022}
}
@inproceedings{Filotheou2022iA,
author={Filotheou, Alexandros and Sergiadis, Georgios D. and Dimitriou, Antonis G.},
booktitle={2022 IEEE/RSJ International Conference on Intelligent Robots and Systems (IROS)},
title={FSM: Correspondenceless scan-matching of panoramic 2D range scans},
month={oct},
year={2022},
pages={6968-6975},
doi={10.1109/IROS47612.2022.9981228}
}
@article{Filotheou2023A,
author = {Filotheou, Alexandros and Symeonidis, Andreas L. and Sergiadis, Georgios D. and Dimitriou, Antonis G.},
doi = {10.1016/j.array.2023.100288},
issn = {25900056},
journal = {Array},
month = {jul},
pages = {100288},
title = {{Correspondenceless scan-to-map-scan matching of 2D panoramic range scans}},
url = {https://linkinghub.elsevier.com/retrieve/pii/S2590005623000139},
volume = {18},
year = {2023}
}
\end{filecontents}
\usepackage{filecontents}
\usepackage{bibentry}


\DeclareBibliographyCategory{nobibliography}
\addtocategory{nobibliography}{Tzitzis2020}
\addtocategory{nobibliography}{Tzitzis2020a}
\addtocategory{nobibliography}{Mylonopoulos2021}
\addtocategory{nobibliography}{Dimitriou2021a}
\addtocategory{nobibliography}{Tzitzis2023}
\addtocategory{nobibliography}{8739423}
\addtocategory{nobibliography}{8739486}
\addtocategory{nobibliography}{Filotheou2020bA}
\addtocategory{nobibliography}{Filotheou2020cA}
\addtocategory{nobibliography}{Filotheou2022eA}
\addtocategory{nobibliography}{Filotheou2022iA}
\addtocategory{nobibliography}{Filotheou2023A}
\addbibresource{papers_during_phd.bib}


% Make author bold
\def\makenamesetup{%
  \def\bibnamedelima{~}%
  \def\bibnamedelimb{ }%
  \def\bibnamedelimc{ }%
  \def\bibnamedelimd{ }%
  \def\bibnamedelimi{ }%
  \def\bibinitperiod{.}%
  \def\bibinitdelim{~}%
  \def\bibinithyphendelim{.-}}
\newcommand*{\makename}[2]{\begingroup\makenamesetup\xdef#1{#2}\endgroup}

\newcommand*{\boldname}[3]{%
  \def\lastname{#1}%
  \def\firstname{#2}%
  \def\firstinit{#3}}
\boldname{}{}{}

% Patch new definitions
\renewcommand{\mkbibnamegiven}[1]{%
  \ifboolexpr{ ( test {\ifdefequal{\firstname}{\namepartgiven}} or test {\ifdefequal{\firstinit}{\namepartgiven}} ) and test {\ifdefequal{\lastname}{\namepartfamily}} }
  {\mkbibbold{#1}}{#1}%
}

\renewcommand{\mkbibnamefamily}[1]{%
  \ifboolexpr{ ( test {\ifdefequal{\firstname}{\namepartgiven}} or test {\ifdefequal{\firstinit}{\namepartgiven}} ) and test {\ifdefequal{\lastname}{\namepartfamily}} }
  {\mkbibbold{#1}}{#1}%
}

\boldname{Filotheou}{Alexandros}{A.}


























\begin{document}

%\pagestyle{plain} % Removes page numbering
\pagenumbering{gobble}

%----------------------------------------------------------------------------------------
%	NAME AND CONTACT INFORMATION
%----------------------------------------------------------------------------------------

\par{\centering{\Huge Αλέξανδρος Φιλοθέου}\bigskip\par}

\section{Στοιχεία Ταυτότητος}

\begin{tabular}{rp{10cm}}
Τόπος $\&$ ημερομηνία γεννήσεως                 & Θεσσαλονίκη | 8 Νοεμβρίου 1987 \\
Τρέχουσα τοποθεσία $\&$ ημερομηνία ανανέωσης CV & Θεσσαλονίκη | Μάιος 2024\\
Τηλέφωνο                                        & 693 87 87 677 \\
e-mail                                          & \href{mailto:alexandros.filotheou@gmail.com}{alexandros.filotheou@gmail.com}
\end{tabular}\\

%----------------------------------------------------------------------------------------
%	WORK EXPERIENCE
%----------------------------------------------------------------------------------------

\section{Εργασιακή Εμπειρία}

\begin{tabular}{rp{12cm}}
%------------------------------------------------
09.2023 - \hfill παρόν \hfill & \textbf{Μεταδιδακτορικός Ερευνητής} \\
                              &  Κέντρο Έρευνας και Τεχνολογικής Ανάπτυξης (ΕΚΕΤΑ), Θεσσαλονίκη\\
&\\
09.2018 - 03.2023 & \textbf{Εργολήπτης Ερευνητικών Έργων Ρομποτικής} \\
                  & Τμήμα Ηλεκτρολόγων Μηχανικών και Μηχανικών Υπολογιστών \\
                  & Αριστοτέλειο Πανεπιστήμιο Θεσσαλονίκης, Θεσσαλονίκη\\
&\\
09.2016 - 11.2016 & \textbf{Teaching Assistant} $\cdot$ DD2380 Artificial Intelligence \\
                  & KTH Royal Institute of Technology, Stockholm, Sweden\\
&\\
%------------------------------------------------
10.2011 - 03.2012 & \textbf{Σχεδιαστής Βάσεων Δεδομένων} \\
                  & Εγνατία Οδός Α.Ε., Θεσσαλονίκη \\
                  & Σχεδιασμός και υλοποίηση ενοποιημένης Βάσης Δεδομένων σε ORACLE 10g για
                    το σύστημα ενόργανης παρακολούθησης κατολισθήσεων και γεωτεχνικών προβλημάτων οδών,
                    στα πλαίσια του Ευρωπαϊκού Ερευνητικού Προγράμματος IRIS. \\
&\\
%------------------------------------------------
03.2011 - 05.2011 & \textbf{Προγραμματιστής Βάσεων Δεδομένων} $\cdot$ Πρακτική Άσκηση \\
                  & Εγνατία Οδός Α.Ε., Θεσσαλονίκη \\
                  & Δημιουργία υποσυστήματος ανάκτησης δεδομένων με χρήση παραμετροποιήσιμων
                    κριτηρίων, από το μητρώο γεφυρών της Εγνατίας Οδού (σύστημα BRIDGE), καθώς και
                    δημιουργία σχετικών αναφορών. Χαρακτηρίστηκε ως η πρώτη ολοκληρωμένη πρακτική
                    εργασία στην Εγνατία Οδό. Η εφαρμογή αναπτύχθηκε με ORACLE Developers Tools
                    (ORACLE Forms, ORACLE Reports), ενώ η βάση δεδομένων που χρησιμοποιήθηκε ήταν η ORACLE $10g$ v. $10.2.0.4.$\\
&\\
%------------------------------------------------
07.2008 - 06.2009 & \textbf{Μηχανικός Τηλεπικοινωνιών} \\
                  & Οργανισμός Τηλεπικοινωνιών Ελλάδος, Θεσσαλονίκη\\
                  & Τεχνική υπηρεσία σε ζητήματα τοπικών δικτύων και δικτύων ευρείας περιοχής. \\
\end{tabular} \\

%----------------------------------------------------------------------------------------
%	VOLUNTARY EXPERIENCE
%----------------------------------------------------------------------------------------

\section{Εθελοντική Εμπειρία}

\begin{tabular}{rp{12cm}}
10.2013 - 07.2014 & \textbf{Μηχανικός Μηχανικής Όρασης, Ομάδα ρομποτικής PANDORA,
                    Τμήμα Ηλεκτρολόγων Μηχανικών και Μηχανικών Υπολογιστών, Αριστοτέλειο Πανεπιστήμιο Θεσσαλονίκης} \\
                  & Σχεδίαση της αρχιτεκτονικής, υλοποίηση και ενδελεχής τεκμηρίωση
                    του συστήματος Εύρεσης Οπών του ρομποτικού πράκτορα PANDORA, χρησιμοποιώντας το ROS, αισθητήρες RBG+D
                    (Microsoft Kinect, ASUS Xtion), στα πλάισια και τις συνθήκες του διεθνούς διαγωνισμού RoboCup Rescue.\\
\end{tabular} \\



%----------------------------------------------------------------------------------------
%	EDUCATION
%----------------------------------------------------------------------------------------

\section{Εκπαίδευση}

\begin{tabular}{rp{11cm}}
09.2018 $-$ 06.2023 & \textbf{Διδακτορικό Δίπλωμα} \\
                    & Τμήμα Ηλεκτρολόγων Μηχανικών και Μηχανικών Υπολογιστών \\
                    & Αριστοτέλειο Πανεπιστήμιο Θεσσαλονίκης \\

&\\                 & \textbf{Τίτλος Διατριβής} $\cdot$ Εκτίμηση στάσης αισθητήρα LIDAR δισδιάστατων μετρήσεων μέσω ευθυγράμμισης πραγματικών με εικονικές σαρώσεις\\
                    & Επιβλέπων: Καθ. Γεώργιος Σεργιάδης, Τομέας Τηλεπικοινωνιών\\
                    & Επιτροπή: Γεώργιος Σεργιάδης (Α.Π.Θ.), Ανδρέας Συμεωνίδης (Α.Π.Θ.), Τραϊανός Γιούλτσης (.ΑΠ.Θ.), Ζωή Δουλγέρη (Α.Π.Θ.), Νικόλαος Φαχαντίδης (ΠΑ.ΜΑΚ), Άγγελος Μπλέτσας (Πολυτεχνείο Κρήτης), Αναστάσιος Ντελόπουλος (Α.Π.Θ.) \\

&\\
09.2015 $-$ 06.2017 & \textbf{Μεταπτυχιακό Δίπλωμα} \\
                    & KTH Royal Institute of Technology, Stockholm, Sweden\\
                    & School of Electrical Engineering and Computer Science\\
                    & Τίτλος Προγράμματος: \textit{Systems, Control, and Robotics}\\

&\\                 & \textbf{Εργασία Πτυχίου} $\cdot$ Εύρωστος Αποκεντρωμένος Έλεγχος Πολλαπλών Συνεργατικών Ρομποτικών Συστημάτων: Μία Ενδο-περιοριστική
                      Προσέγγιση Υποχωρώντος Ορίζοντος \\
                    & Επιβλέπων: Καθ. Δήμος Δημαρόγκωνας, Τμήμα Αυτομάτου Ελέγχου\\
&\\
09.2005 $-$ 07.2013 & \textbf{Δίπλωμα Ηλεκτρολόγου Μηχανικού \& Μηχανικού Υπολογιστών} \\
                    & Αριστοτέλειο Πανεπιστήμιο Θεσσαλονίκης \\
                    & Βαθμός: 7.94 / 10.0, Κατάταξη: 23 / 280 \\
&\\
                    & \textbf{Διπλωματική Εργασία} $\cdot$ Πολυκατηγορική Ταξινόμηση με Μανθάνοντα Συστήματα Ταξινομητών\\
                    & Επιβλέπων: Καθ. Περικλής Μήτκας, Τομέας Ηλεκτρονικής και Υπολογιστών\\
                    & Επιτροπή: Περικλής Μήτκας (ΑΠΘ), Αναστάσιος Ντελόπουλος (ΑΠΘ), Ανδρέας Συμεωνίδης (ΑΠΘ) \\
\end{tabular}\\



%----------------------------------------------------------------------------------------
%	Δημοσιεύσεις
%----------------------------------------------------------------------------------------
\section{Δημοσιεύσεις}

\href{https://scholar.google.com/citations?view\_op=list\_works\&hl=en\&user=9\_hI4hMAAAAJ}{Σύνδεσμος προς Google Scholar}\\

{
\fullcite{Tzitzis2023}\\

\fullcite{Filotheou2023A}\\

\fullcite{Filotheou2022iA}\\

\fullcite{Filotheou2022}\\

\fullcite{Filotheou2022eA}\\

\fullcite{Mylonopoulos2021}\\

\fullcite{Dimitriou2021a}\\

\fullcite{Filotheou2020cA}\\

\fullcite{Tzitzis2020}\\

\fullcite{Tzitzis2020a}\\

\fullcite{Filotheou2020bA}\\

\fullcite{8739423}\\

\fullcite{8739486} \\

\fullcite{Filotheou2020} \\

\fullcite{Filotheou2018}
}


%----------------------------------------------------------------------------------------
%	Distinctions
%----------------------------------------------------------------------------------------

\section{Διακρίσεις}
\begin{tabular}{rp{14cm}}

$2016$ & Βοηθός Διδασκαλίας, DD2380 - Artificial Intelligence, \\ & υπό την επίβλεψη του καθηγητού Patric Jensfelt, KTH Royal Institute of Technology, Σουηδία\\

%------------------------------------------------

$2015$ & $2^{\eta}$ θέση στην κλάση Αυτονομίας στο διαγωνισμό RoboCup Rescue ως μέλος της ομάδας ρομποτικής PANDORA \\

%------------------------------------------------

$2013$ & $30^{o\varsigma}$ εκ των $224$ φοιτητων που αποφοίτησαν το $2013$ από το Tμήμα Ηλεκτρολόγων Μηχανικών και Μηχανικών Υπολογιστών, Α.Π.Θ. \\

%------------------------------------------------

$2011$ & Top of class στο μάθημα Βάσεις Δεδομένων, χειμερινό εξάμηνο $2010 - 2011$, Α.Π.Θ. \\

%------------------------------------------------

$2005$ & $21^{o\varsigma}$ εκ των $280$ μαθητών που πέρασαν στο Tμήμα Ηλεκτρολόγων Μηχανικών και Μηχανικών Υπολογιστών, Α.Π.Θ. το $2005$\\
&\\
\end{tabular}





%----------------------------------------------------------------------------------------
%	COMPUTER SKILLS
%----------------------------------------------------------------------------------------
\section{Γνώσεις Υπολογιστών}

\begin{tabular}{rp{9cm}}
  Γλώσσες & \texttt{C/C++}, \texttt{MATLAB}, $\{$\texttt{PL/}$\}$\texttt{SQL}, \texttt{Java}, \texttt{Python}, \texttt{shell}, \texttt{Assembly}\\
&\\
  $\{$Μετα-$\}$λειτουργικά Συστήματα & \texttt{Linux}, \texttt{ROS 2}, \texttt{ROS}  \\
&\\
  Προγράμματα Γραφικών & \texttt{AutoCAD}, \texttt{Gimp}\\
&\\
  Εργαλεία & \texttt{git}, \texttt{Docker}, \texttt{OpenCV}, \LaTeX, \texttt{Oracle Forms} / \texttt{Reports}, \texttt{Microsoft} $\{$\texttt{Visio}, \texttt{Project}, \texttt{Office}$\}$
\end{tabular}


%----------------------------------------------------------------------------------------
%	LANGUAGES
%----------------------------------------------------------------------------------------

\section{Γλώσσες}

\begin{tabular}{rp{12cm}}
Αγγλικά & Fluent - IELTS Score 8.5\\
Ελληνικά & Μητρική
\end{tabular}

%----------------------------------------------------------------------------------------
%	LINKS
%----------------------------------------------------------------------------------------
\section{Σύνδεσμοι}

Ενδεικτικά πακέτα ROS [\href{https://github.com/li9i}{\texttt{github}}]:
\href{https://github.com/li9i/cbgl}{\texttt{cbgl}} $\cdot$
\href{https://github.com/li9i/fsm\_lidom\_ros}{\texttt{fsm\_lidom\_ros}} $\cdot$
\href{https://github.com/li9i/lama\_odom}{\texttt{lama\_odom}} $\cdot$
\href{https://github.com/li9i/pandora\_vision\_2014/tree/hydro-devel/pandora\_vision\_hole\_detector}{\texttt{pvhd}}

Demos / videos: \href{https://www.youtube.com/watch?v=xaDKjI0WkDc}{\texttt{cbgl}} $\cdot$ \href{https://cultureid.web.auth.gr/?page\_id=200&lang=en}{\texttt{cultureid}} $\cdot$ \href{https://www.youtube.com/watch?v=hB4qsHCEXGI}{\texttt{fsm}} $\cdot$ \href{https://relief.web.auth.gr/}{\texttt{relief}} $\cdot$ \href{https://www.youtube.com/watch?v=937OZez1iN8}{\texttt{mpc}}
\\

%----------------------------------------------------------------------------------------
%	References
%----------------------------------------------------------------------------------------
\section{Συστάσεις}
\noindent Δρ. Αντώνης Γ. Δημητρίου $\cdot$ Συντονιστής των έργων στα οποία εργάσθηκα στο Α.Π.Θ. \\
\faPhone \ 6978896350 $\cdot$ \faEnvelopeO \ \href{mailto:antodimi@auth.gr}{antodimi@auth.gr} \\


\end{document}

\documentclass[a4paper,10pt,twoside]{article}
\usepackage{fullpage}
\usepackage{fontspec}
\usepackage[utf8]{inputenc}

%\defaultfontfeatures{Mapping=tex-text}
%\setmainfont[Mapping=tex-text]{GFS Didot} % Main document font
%\setmainfont{cmr12} % Main document font
\setmainfont{GFSdidot} % Main document font

\usepackage[usenames,dvipsnames]{xcolor} % Required for specifying custom colors

%\usepackage[big]{layaureo} % Margin formatting of the A4 page, an alternative to layaureo can be
\usepackage{fullpage}
% To reduce the height of the top margin uncomment:
\addtolength{\voffset}{-1.3cm}

\usepackage{hyperref} % Required for adding links	and customizing them
\definecolor{linkcolour}{rgb}{0,0.2,0.6} % Link color
\hypersetup{colorlinks,breaklinks,urlcolor=linkcolour,linkcolor=linkcolour} % Set link colors throughout the document

\usepackage{titlesec} % Used to customize the \section command
\titleformat{\section}{\Large\scshape\raggedright}{}{0em}{}[\titlerule] % Text formatting of sections
\titlespacing{\section}{0pt}{3pt}{3pt} % Spacing around sections


\usepackage{fancyhdr}
\setlength{\headheight}{15.2pt}
\setlength{\textheight}{710pt}
\pagestyle{fancy}
\fancyhead[]{}
\renewcommand{\headrulewidth}{0pt}
\fancyfoot[LO,RE]{Βιογραφικό Σημείωμα $-$ Αλέξανδρος Φιλοθέου}

\usepackage{tikz}

\begin{document}

%\pagestyle{plain} % Removes page numbering
\pagenumbering{gobble}

%----------------------------------------------------------------------------------------
%	NAME AND CONTACT INFORMATION
%----------------------------------------------------------------------------------------

\par{\centering{\Huge Αλέξανδρος Φιλοθέου}\bigskip\par}

\section{Στοιχεία Ταυτότητος}

\begin{tabular}{rp{10cm}}
Τόπος $\&$ ημερομηνία γεννήσεως 	& Θεσσαλονίκη | 8 Νοεμβρίου 1987 \\
Διεύθυνση           			        & Αγνώστου Στρατιώτου 10, ΤΚ 54631 Θεσσαλονίκη \\
Τηλέφωνο                          &  693 87 87 677 \\
e-mail                            & \href{mailto:alexandros.filotheou@gmail.com}{alexandros.filotheou@gmail.com}
\end{tabular}\\

%----------------------------------------------------------------------------------------
%	EDUCATION
%----------------------------------------------------------------------------------------

\section{Εκπαίδευση}

\begin{tabular}{rp{11cm}}
03.2020 $-$ παρόν   & \textbf{Υποψήφιος Διδάκτορας} \\
                    & Τμήμα Ηλεκτρολόγων Μηχανικών και Μηχανικών Υπολογιστών \\
                    & Αριστοτέλειο Πανεπιστήμιο Θεσσαλονίκης \\
&\\
09.2015 $-$ 06.2017 & \textbf{Μεταπτυχιακό Δίπλωμα Ειδίκευσης} \\
                    & KTH Royal Institute of Technology, Στοκχόλμη, Σουηδία\\
                    & Τίτλος Προγράμματος: \textit{Systems, Control and Robotics},
                      School of Electrical Engineering and Computer Science, KTH Royal Institute of Technology\\

&\\                 & \textbf{Εργασία Πτυχίου} ``Εύρωστος Αποκεντρωμένος Έλεγχος Πολλαπλών Συνεργατικών Ρομποτικών Συστημάτων: Μία Ενδο-περιοριστική
                      Προσέγγιση Υποχωρώντος Ορίζοντος" \\
                    & Επιβλέπων: Καθ. Δήμος Δημαρόγκωνας, Τμήμα Αυτομάτου Ελέγχου, Σχολή Ηλεκτρολόγων Μηχανικών και Επιστήμης των Υπολογιστών,
                      KTH Royal Institute of Technology\\
&\\
09.2005 $-$ 07.2013 & \textbf{Δίπλωμα Ηλεκτρολόγου Μηχανικού \& Μηχανικού Υπολογιστών} \\
                    & Αριστοτέλειο Πανεπιστήμιο Θεσσαλονίκης \\
                    & Βαθμός: 7.94 / 10.0, Κατάταξη: 23 / 280 \\
&\\
                    & \textbf{Διπλωματική Εργασία} ``Πολυκατηγορική Ταξινόμηση με Μανθάνοντα Συστήματα Ταξινομητών"\\
                    & Επιβλέπων: Καθ. Περικλής Μήτκας, Τομέας Ηλεκτρονικής και Υπολογιστών, Τμήμα Ηλεκτρολόγων Μηχανικών,
                      Αριστοτέλειο Πανεπιστήμιο Θεσσαλονίκης\\
                    & Βαθμός: 10.0 / 10.0
\end{tabular}\\


%----------------------------------------------------------------------------------------
%	WORK EXPERIENCE
%----------------------------------------------------------------------------------------

\section{Εργασιακή Εμπειρία}

\begin{tabular}{rp{12cm}}
%------------------------------------------------
09.2018 - παρόν   & \textbf{Ερευνητής Ρομποτικής και Ελέγχου}, Τμήμα Ηλεκτρολόγων Μηχανικών και Μηχανικών Υπολογιστών, Θεσσαλονίκη, Ελλάδα\\
&\\
09.2016 - 11.2016 & \textbf{Βοηθός Διδασκαλίας} για το μάθημα DD2380 $-$ Τεχνητή Νοημοσύνη, KTH Royal Institute of Technology, Στοκχόλμη, Σουηδία\\
&\\
%------------------------------------------------
10.2011 - 03.2012 & \textbf{Σχεδιαστής Βάσεων Δεδομένων}, Εγνατία Οδός Α.Ε., Θεσσαλονίκη \\
                  & Σχεδιασμός και υλοποίηση ενοποιημένης Βάσης Δεδομένων σε ORACLE 10g για
                    το σύστημα ενόργανης παρακολούθησης κατολισθήσεων και γεωτεχνικών προβλημάτων οδών,
                    στα πλαίσια του Ευρωπαϊκού Ερευνητικού Προγράμματος IRIS. \\
&\\
%------------------------------------------------
03.2011 - 05.2011 & \textbf{Προγραμματιστής}, Πρακτική Άσκηση, Εγνατία Οδός Α.Ε., Θεσσαονίκη \\
                  & Δημιουργία υποσυστήματος ανάκτησης δεδομένων με χρήση παραμετροποιήσιμων
                    κριτηρίων, από το μητρώο γεφυρών της Εγνατίας Οδού (σύστημα BRIDGE), καθώς και
                    δημιουργία σχετικών αναφορών. Χαρακτηρίστηκε ως η πρώτη ολοκληρωμένη πρακτική
                    εργασία στην Εγνατία Οδό. Η εφαρμογή αναπτύχθηκε με ORACLE Developers Tools
                    (ORACLE Forms, ORACLE Reports), ενώ η βάση δεδομένων που χρησιμοποιήθηκε ήταν η ORACLE $10g$ v. $10.2.0.4.$\\
&\\
%------------------------------------------------
07.2008 - 06.2009 & \textbf{Βοηθός Τεχνικής Υποστήριξης} Ο.Τ.Ε., Θεσσαλονίκη\\
                  & Τεχνική υποστήριξη και εξυπηρέτηση πελατών σε θέματα Internet
                    και συνδέσεων ADSL. Αποκόμιση εμπειρίας σε δίκτυα υπολογιστών
                    και καλλιέργεια επικοινωνιακών ικανοτήτων.\\
\end{tabular} \\


%----------------------------------------------------------------------------------------
%	VOLUNTARY EXPERIENCE
%----------------------------------------------------------------------------------------

\section{Εθελοντική Εμπειρία}

\begin{tabular}{rp{12cm}}
10.2013 - 07.2014 & \textbf{Μηχανικός Μηχανικής Όρασης, Ομάδα ρομποτικής PANDORA, Εργαστήριο Ρομποτικής,
                    Τμήμα Ηλεκτρολόγων Μηχανικών και Μηχανικών Υπολογιστών, Αριστοτέλειο Πανεπιστήμιο Θεσσαλονίκης} \\
                  & Σχεδίαση της αρχιτεκτονικής, υλοποίηση και ενδελεχής τεκμηρίωση
                    του συστήματος Εύρεσης Οπών του ρομποτικού πράκτορα PANDORA, χρησιμοποιώντας το ROS, αισθητήρες RBG+D
                    (Microsoft Kinect, ASUS Xtion), στα πλάισια και τις συνθήκες του διεθνούς διαγωνισμού RoboCup Rescue.\\
\end{tabular} \\

%----------------------------------------------------------------------------------------
%	Δημοσιεύσεις
%----------------------------------------------------------------------------------------
\section{Δημοσιεύσεις}

A. Filotheou, Emmanouil Tsardoulias, Antonis Dimitriou, Andreas Symeonidis and Loukas Petrou
Pose Selection and Feedback Methods in Tandem Combinations of Particle Filters with Scan-Matching for 2D Mobile Robot Localisation
\textit{Journal of Intelligent \& Robotic Systems, 2020}

A. Filotheou, Emmanouil Tsardoulias, Antonis Dimitriou, Andreas Symeonidis and Loukas Petrou
Quantitative and Qualitative Evaluation of ROS-Enabled Local and Global Planners in 2D Static Environments.
\textit{Journal of Intelligent \& Robotic Systems, 2019}

A. Filotheou, A. Nikou, and D. V. Dimarogonas. Robust Decentralized Navigation
of Multi-Agent Systems with Collision Avoidance and Connectivity Maintenance
Using Model Predictive Controllers. \textit{International Journal of Control, 2018}

A. Filotheou, A. Nikou, and D. V. Dimarogonas. Decentralized Control of Uncertain
Multi-Agent Systems with Connectivity Maintenance and Collision Avoidance.
\textit{IEEE European Control Conference (ECC), 2018}





%----------------------------------------------------------------------------------------
%	COMPUTER SKILLS
%----------------------------------------------------------------------------------------
\section{Γνώσεις Υπολογιστών}

\begin{tabular}{rp{9cm}}
Γλώσσες & C/C++, MATLAB, $\{$PL/$\}$SQL, Java, Python, Shell Scripting, Assembly \\
&\\
$\{$Μετα-$\}$λειτουργικά Συστήματα & Linux, ROS \\
&\\
Προγράμματα Γραφικών & AutoCAD, Gimp\\
&\\
Εργαλεία & UML, Git, OpenCV, \LaTeX, Apache HTTP Server, AVR studio,
   Oracle Forms / Reports, OpenOffice, PSpice
   Microsoft $\{$Visio, Project, Office$\}$
\end{tabular}


%----------------------------------------------------------------------------------------
%	LANGUAGES
%----------------------------------------------------------------------------------------

\section{Γλώσσες}

\begin{tabular}{rp{12cm}}
Αγγλικά & Fluent - IELTS Score 8.5 (2014) \\
Γαλλικά & Basic Knowledge - Delf A2 (2001) \\
Ελληνικά & Μητρική
\end{tabular}


%----------------------------------------------------------------------------------------
%	LINKS
%----------------------------------------------------------------------------------------

\section{Σύνδεσμοι}
code: \url{https://github.com/li9i}

demo: \url{https://www.youtube.com/user/canhartt/videos}

\end{document}
